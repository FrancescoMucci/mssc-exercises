\chapter{Sistemi Concorrenti}
\label{chap:parte3}

\section*{Esercizio 11.8}
\phantomsection
\addcontentsline{toc}{section}{Esercizio 11.8}
\label{es:11.8}

\begin{tcolorbox} \cite{mssc2016}
Dimostrare che l'unione di tutte le bisimulazioni di branching è: 
\begin{enumerate}
\item una bisimulazione di branching;
\item un'equivalenza.
\end{enumerate}
\end{tcolorbox}

Per provare che l'unione di tutte le bisimulazioni di branching è una bisimulazione di branching, dobbiamo prima dimostrare il seguente lemma.
\begin{customlemma}{11.8.1.1}[L'unione preserva le bisimulazioni di branching]
\label{lemma:11.8.1.1} \hfill \\
Sia $\tupla{Q, A_{\tau}, \rightarrow}$ un LTS e sia ogni $S_i$ con $i \in \{1, \ldots, n\}$ una bisimulazione di branching su $Q$, possiamo provare che $\bigcup_{i=1}^{n} S_i$ è a sua volta una bisimulazione di branching.
\end{customlemma}
\begin{proof}
Vogliamo quindi mostrare che:
\begin{enumerate}
\item $\forall\ p,q \in Q \valeche \tupla{p,q} \in \bigcup_{i=1}^{n} S_i 
\implies \tupla{q,p} \in \bigcup_{i=1}^{n} S_i$;
\item $\forall\ \tupla{p,q} \in \bigcup_{i=1}^{n} S_i,\ \forall\ \mu \in A_{\tau} \valeche
\forall\ p' \in Q \tc p \xrightarrow{\mu} p'$, almeno una delle seguenti condizioni deve essere soddisfatta:
\begin{enumerate}
\item $\mu = \tau \myland \tupla{p', q} \in \bigcup_{i=1}^{n} S_i$;
\item $\exists\ q_x, q' \in Q \tc q \xRightarrow{\varepsilon} q_x \xrightarrow{\mu} q' 
	\con \tupla{p, q_x} \in \bigcup_{i=1}^{n} S_i 
	\myland \tupla{p', q'} \in \bigcup_{i=1}^{n} S_i$.
\end{enumerate}
\end{enumerate}
Consideriamo una qualunque coppia $\tupla{p,q} \in \bigcup_{i=1}^{n} S_i$, per definizione di unione tra insiemi abbiamo che: \[
	\tupla{p,q} \in \bigcup_{i=1}^{n} S_i 
	\implies \exists\ k \in \{1, \ldots, n\} \tc \tupla{p,q} \in S_k.
\]
\begin{enumerate}[leftmargin=*]
\item Essendo che $S_k$ è simmetrica risulta \[
	\tupla{p,q} \in S_k \implies \tupla{q,p} \in S_k.
\]
Inoltre, dato che $S_k \subseteq \bigcup_{i=1}^{n} S_i$, possiamo affermare che \[
	\tupla{q,p} \in S_k \implies \tupla{q,p} \in \bigcup_{i=1}^{n} S_i;
\]
\item Consideriamo un qualsiasi $p' \in Q \tc p \xrightarrow{\mu} p' \con \mu \in A_{\tau}$.
\begin{enumerate}
\item Se $\mu = \tau$, essendo che $S_k$ è bisimulazione di branching, sappiamo che \[
	\tupla{p,q} \in S_k \myland p \xrightarrow{\tau} p'
	\implies \tupla{p',q} \in S_k;
\]
Dato che $S_k \subseteq \bigcup_{i=1}^{n}$, risulta che \[
	\tupla{p',q} \in S_k \implies \tupla{p',q} \in \bigcup_{i=1}^{n} S_i.
\]
\item Se $\mu \neq \tau$, essendo che $S_k$ è bisimulazione di branching, risulta che
\begin{align*}
	&\tupla{p,q} \in S_k \myland p \xrightarrow{\mu} p' \implies \\
	&\tab \exists\ q_x, q' \in Q \tc q \xRightarrow{\varepsilon} q_x \xrightarrow{\mu} q' 
	\con \tupla{p, q_x} \in S_k \myland \tupla{p', q'} \in S_k
\end{align*}
Dato che $S_k \subseteq \bigcup_{i=1}^{n} S_i$, possiamo concludere che
\begin{align*}
	&\tupla{p, q_x} \in S_k  \implies \tupla{p, q_x} \in \bigcup_{i=1}^{n} S_i \\
	&\tupla{p', q'} \in S_k  \implies \tupla{p', q'} \in \bigcup_{i=1}^{n} S_i
\end{align*}
\end{enumerate}
\end{enumerate}
\end{proof}

\begin{customthm}{11.8.1}[$\approx_b$ è la più grande bisimulazione di branching]
\label{th:11.8.1}
La bisimilarità di branching \[
	\approx_b \triangleq \bigcup\ \{R \mid R\ \text{è una bisimulazione di branching}\}
\]
è una bisimulazione di branching.
\end{customthm}
\begin{proof}
Per il lemma \ref{lemma:11.8.1.1} sappiamo che l'unione preserva le bisimulazioni di branching; di conseguenza, risulta immediato che l'unione di tutte le bisimulazioni di branching sia a sua volta una bisimulazione di branching.
\end{proof}

Per provare che l'unione di tutte le bisimulazioni di branching è un'equivalenza dovremo mostrare che questa è riflessiva, simmetrica e transitiva; per fare ciò dovremo, in ordine, sfruttare i risultati dei seguenti lemmi.
\begin{customlemma}{11.8.2.1}[L'identità è una bisimulazione di branching]
\label{lemma:11.8.2.1} \hfill \\
Sia $\tupla{Q, A_{\tau}, \rightarrow}$ un LTS e $Id_Q$ la relazione identità su $Q$, possiamo provare che $Id_Q$ è una bisimulazione di branching.
\end{customlemma}
\begin{proof}
Vogliamo quindi mostrare che:
\begin{enumerate}
\item $Id_Q$ è simmetrica;
\item $\forall\ \tupla{p,p} \in Id_Q,\ \forall\ \mu \in A_{\tau} \valeche
\forall\ p' \in Q \tc p \xrightarrow{\mu} p'$, almeno una delle seguenti condizioni deve essere soddisfatta:
\begin{enumerate}
\item $\mu = \tau \myland \tupla{p', p} \in Id_Q$;
\item $\exists\ q_x, q' \in Q \tc p \xRightarrow{\varepsilon} q_x \xrightarrow{\mu} q' 
	\con \tupla{p, q_x} \in Id_Q 
	\myland \tupla{p', q'} \in Id_Q$.
\end{enumerate}
\end{enumerate}
Consideriamo una qualunque coppia $\tupla{p,p} \in Id_Q$.
\begin{enumerate}[leftmargin=*]
\item $Id_Q$ risulta simmetrico per definizione.
\item Consideriamo un qualsiasi $p' \in Q \tc p \xrightarrow{\mu} p' \con \mu \in A_{\tau}$.
\begin{enumerate}
\item La prima condizione risulta soddisfatta solo nel caso in cui $p \equiv p'$, in tal caso avremo infatti che: \[
	\tupla{p,p} \in Id_Q \myland p \xrightarrow{\tau} p
\]
\item La seconda condizione risulta sempre soddisfatta, ci basta considerare $p \equiv q_x$ e $p' \equiv q'$ ed avremo che:
\begin{align*}
	&\tupla{p,p} \in Id_Q \myland p \xrightarrow{\mu} p' \implies \\
	&\tab p \xRightarrow{\varepsilon} p \xrightarrow{\mu} p' 
	\con \tupla{p, p} \in Id_Q \myland \tupla{p', p'} \in Id_Q
\end{align*}
\end{enumerate}
\end{enumerate}
\end{proof}

\begin{customlemma}{11.8.2.2}[L'inversione preserva le bisimulazioni di branching]
\label{lemma:11.8.2.2} \hfill \\
Sia $\tupla{Q, A_{\tau}, \rightarrow}$ un LTS e sia $S$ una bisimulazione di branching su $Q$, possiamo provare che $S^{-1}$ è a sua volta una bisimulazione di branching \footnote{\textit{Dimostrazione alternativa.} Dato che $S$ è bisimulazione di branching, sappiamo che $S$ è simmetrica; inoltre, per definizione di relazione simmetrica, abbiamo che $S=S^{-1}$ e, per tale motivo, possiamo concludere che anche $S^{-1}$ è bisimulazione di branching.}. 
\end{customlemma}
\begin{proof}
Vogliamo quindi mostrare che:
\begin{enumerate}
\item $\forall\ q,p \in Q \valeche \tupla{q,p} \in S^{-1} 
\implies \tupla{p,q} \in S^{-1}$;
\item $\forall\ \tupla{q,p} \in S^{-1}\, \forall\ \mu \in A_{\tau} \valeche
\forall\ q' \in Q \tc q \xrightarrow{\mu} q'$, almeno una delle seguenti condizioni deve essere soddisfatta:
\begin{enumerate}
\item $\mu = \tau \myland \tupla{q', p} \in S^{-1}$;
\item $\exists\ p_x, p' \in Q \tc p \xRightarrow{\varepsilon} p_x \xrightarrow{\mu} p' 
	\con \tupla{q, p_x} \in S^{-1} 
	\myland \tupla{q', p'} \in S^{-1}$.
\end{enumerate}
\end{enumerate}
Consideriamo una qualunque coppia $\tupla{q,p} \in S^{-1}$, per definizione di inversa abbiamo che \[
	\tupla{q,p} \in S^{-1} \implies \tupla{p,q} \in S.
\]
Essendo che $S$ è per ipotesi simmetrica, sappiamo anche che
\[
	\tupla{p,q} \in S \implies \tupla{q,p} \in S.
\]
\begin{enumerate}[leftmargin=*]
\item Per definizione di inversa sappiamo che
\[
	\tupla{q,p} \in S \implies \tupla{p,q}  \in S^{-1}.
\]
\item Consideriamo un qualsiasi $q' \in Q \tc q \xrightarrow{\mu} q' \con \mu \in A_{\tau}$.
\begin{enumerate}
\item Se $\mu = \tau$, essendo che $S$ è bisimulazione di branching, sappiamo per certo che \[
	\tupla{q,p} \in S \myland q \xrightarrow{\tau} q'
	\implies \tupla{q',p} \in S;
\]
Dato che $S$ è simmetrica, risulta che \[
	\tupla{q',p} \in S \implies \tupla{p,q'} \in S.
\]
Per definizione di inversa, concludiamo che \[
	\tupla{p,q'} \in S \implies \tupla{q',p} \in S^{-1}.
\]
\item Se $\mu \neq \tau$, essendo che $S$ è bisimulazione di branching, risulta che
\begin{align*}
	&\tupla{q,p} \in S \myland q \xrightarrow{\mu} q' \implies \\
	&\tab \exists\ p_x, p' \in Q \tc p \xRightarrow{\varepsilon} p_x \xrightarrow{\mu} p' 
	 \con \tupla{q, p_x} \in S \myland \tupla{q', p'} \in S
\end{align*}
Dato che $S$ è simmetrica, sappiamo che
\begin{align*}
	&\tupla{q, p_x} \in S \implies \tupla{p_x, q} \in S; \\
	&\tupla{q', p'} \in S \implies \tupla{p', q'} \in S.
\end{align*}
Per definizione di inversa, possiamo concludere che
\begin{align*}
	&\tupla{p_x, q} \in S  \implies \tupla{q, p_x} \in S^{-1};\\
	&\tupla{p', q'} \in S  \implies \tupla{q', p'} \in S^{-1}.
\end{align*}
\end{enumerate}
\end{enumerate}
\end{proof}

\begin{customlemma}{11.8.2.3}[La composizione preserva le bisimulazioni di branching]
\label{lemma:11.8.2.3} \hfill \\
Sia $\tupla{Q, A_{\tau}, \rightarrow}$ un LTS e siano $S_1$ e $S_2$ due bisimulazioni di branching su $Q$, possiamo provare che $S_1 \cdot S_2$ è a sua volta una bisimulazione di branching.
\end{customlemma}
\begin{proof}
Diversamente da quanto fatto precedentemente \footnote{Sapere che $S_1$ e $S_2$ sono simmetriche non basta a mostrare che $S_1 \cdot S_2$ è a sua volta simmetrica; per mostrare ciò dovremmo anche sapere che $S_1 \cdot S_2 = S_2 \cdot S_1$.}, per provare che $S_1 \cdot S_2$ è una bisimulazione di branching faremo vedere che $\forall\ \tupla{p,r} \in S_1 \cdot S_2,\ \forall\ \mu \in A_{\tau}$ risulta che:
\begin{enumerate}
\item \textit{$r$ branching simula $p$}:
$\ \forall\ p' \in Q \tc p \xrightarrow{\mu} p'$, almeno una delle seguenti condizioni deve essere soddisfatta:
\begin{enumerate}
\item $\mu = \tau \myland \tupla{p', r} \in S_1 \cdot S_2$;
\item $\exists\ r_x, r' \in Q \tc r \xRightarrow{\varepsilon} r_x \xrightarrow{\mu} r' 
	\con \tupla{p, r_x} \in S_1 \cdot S_2 
	\myland \tupla{p', r'} \in S_1 \cdot S_2$.
\end{enumerate}
\item \textit{$p$ branching simula $r$}:
$\ \forall\ r' \in Q \tc r \xrightarrow{\mu} r'$, almeno una delle seguenti condizioni deve essere soddisfatta:
\begin{enumerate}
\item $\mu = \tau \myland \tupla{r', p} \in S_1 \cdot S_2$;
\item $\exists\ p_x, p' \in Q \tc p \xRightarrow{\varepsilon} p_x \xrightarrow{\mu} p' 
	\con \tupla{r, p_x} \in S_1 \cdot S_2 
	\myland \tupla{r', p'} \in S_1 \cdot S_2$.
\end{enumerate}
\end{enumerate}
\begin{enumerate}[leftmargin=*]
\item Consideriamo una qualunque coppia $\tupla{p,r} \in S_1 \cdot S_2$ ed un qualsiasi $p' \in Q \tc p \xrightarrow{\mu} p' \con \mu \in A_{\tau}$.
Per definizione di composizione, abbiamo che \[
 	\tupla{p,r} \in S_1 \cdot S_2 
 	\implies \exists\ q \in Q \tc \tupla{p,q} \in S_1 \myland \tupla{q, r} \in S_2.
\]
\begin{enumerate}
\item Se $\mu = \tau$, essendo che $S_1$ è bisimulazione di branching, sappiamo per certo che \[
	\tupla{p,q} \in S_1 \myland p \xrightarrow{\tau} p'
	\implies \tupla{p',q} \in S_1.
\]
Per definizione di composizione, possiamo concludere che \[
	\tupla{p',q} \in S_1 \myland \tupla{q, r} \in S_2 
	\implies \tupla{r', p} \in S_1 \cdot S_2.
\]
\begin{figure}
\centering
\begin{tikzpicture}
[node distance = 2cm]
\node[state] (p) {$p$};
\node[state, below right of=p] (q) {$q$};
\node[state, below of=q] (r) {$r$};
\node[state, above right of=q] (p') {$p'$};
\path[-, semithick, right]
	(p) edge[-Stealth, thick, above] node {$\tau$} (p')
	(p) edge[blue] node {$\in S_1$} (q)
	(p') edge[blue] node {$\in S_1$} (q)
	(q) edge[red] node {$\in S_2$} (r);
\end{tikzpicture}
\caption{Composizione di bisimulazioni di branching, caso a).} \label{fig:11.8.2.3.1}
\end{figure}
Graficamente abbiamo la situazione illustrata nella figura \ref{fig:11.8.2.3.1}.
\item Se $\mu \neq \tau$, essendo che $S_1$ è bisimulazione di branching, risulta che
\begin{align*}
	&\tupla{p,q} \in S_1 \myland p \xrightarrow{\mu} p' \implies \\
	&\tab \exists\ q_x, q' \in Q \tc q \xRightarrow{\varepsilon} q_x \xrightarrow{\mu} q' 
	\con \tupla{p, q_x} \in S_1 \myland \tupla{p', q'} \in S_2
\end{align*}
Per definizione di relazione di transizione debole, sappiamo che \[
	q \xRightarrow{\varepsilon} q_x
	\implies \exists\ n \in \N \tc q (\xrightarrow{\tau})^n q_x.
\]
Lavorando per induzione su $n$, lunghezza della sequenza di azioni invisibili eseguite a partire da $q$ per arrivare in $q_x$, potremmo facilmente dimostrare \footnote{Per la dimostrazione completa si veda la proposizione \ref{prop:11.8.2.3.1} presente in appendice.} che \[
	\tupla{q, r} \in S_2 \myland q \xRightarrow{\varepsilon} q_x 
	\implies \tupla{q_x, r} \in S_2.
\]
A questo punto sappiamo che $\tupla{q_x, r} \in S_2$ e che $q_x \xrightarrow{\mu} q'$ con $\mu \neq \tau$, ma allora, essendo che $S_2$ è bisimulazione di branching, risulta che
\begin{align*}
	&\tupla{q_x, r} \in S_2 \myland q_x \xrightarrow{\mu} q'\implies \\  
	&\tab \exists\ r_x, r' \in Q \tc r \xRightarrow{\varepsilon} r_x \xrightarrow{\mu} r' 
		 \con \tupla{q_x, r_x} \in S_2 \myland \tupla{q', r'} \in S_2.
\end{align*}
Per definizione di composizione di relazioni, possiamo quindi concludere che
\begin{align*}
	&\tupla{p, q_x} \in S_1\ \myland \tupla{q_x, r_x} \in S_2  
	  \implies \tupla{p, r_x} \in S_1 \cdot S_2 \\
	&\tupla{p', q'} \in S_1 \myland \tupla{q', r'} \in S_2 
	  \implies \tupla{p', r'} \in S_1 \cdot S_2.
\end{align*}
\begin{figure}
\centering
\begin{tikzpicture}
[node distance = 2cm]
\node[state] (q) {$q$};
\node[state, right of=q] (qx) {$q_x$};
\node[state, right of=qx] (q') {$q'$};
\node[state, above of=qx] (p) {$p$};
\node[state, above of=q'] (p') {$p'$};
\node[state, below of=q] (r) {$r$};
\node[state, below of=qx] (rx) {$r_x$};
\node[state, below of=q'] (r') {$r'$};
\path[-Stealth, thick, above]
	(p) edge node {$\mu \neq \tau$} (p')
	(qx) edge node {$\mu \neq \tau$} (q')
	(rx) edge node {$\mu \neq \tau$} (r')
	(q) edge[-Implies, double] node {$\varepsilon$} (qx)
	(r) edge[-Implies, double] node {$\varepsilon$} (rx);
\path[-, semithick, right]
	(p) edge[blue] node {$\in S_1$} (q)
	(p) edge[blue] node {$\in S_1$} (qx)
	(p') edge[blue] node {$\in S_1$} (q')
	(q) edge[red] node {$\in S_2$} (r)
	(qx) edge[red] node {$\in S_2$} (r)
	(qx) edge[red] node {$\in S_2$} (rx)
	(q') edge[red] node {$\in S_2$} (r');
\end{tikzpicture}
\caption{Composizione di bisimulazioni di branching, caso b).} \label{fig:11.8.2.3.2}
\end{figure}
Graficamente abbiamo la situazione illustrata nella figura \ref{fig:11.8.2.3.2}.
\end{enumerate}
\item Per provare che \textit{$p$ branching simula $r$} procediamo in modo totalmente identico a quanto fatto per provare che \textit{$r$ branching simula $p$}.
\end{enumerate}
\end{proof}

\begin{customthm}{11.8.2}[$\approx_b$ è un'equivalenza]
\label{th:11.8.2}
La bisimilarità di branching $\approx_b$ è una relazione d'equivalenza.
\end{customthm}
\begin{proof}
Vogliamo quindi provare che $\approx_b$ è:
\begin{enumerate}
\item riflessiva: 
	$\forall\ p \in Q \valeche p \approx_b p$; 
\item simmetrica:
	$\forall\ p,q \in Q \valeche
		p \approx_b q
		\implies q \approx_b p$;
\item transitiva:
	$\forall\ p,q,r \in Q \valeche
		p \approx_b q \myland q \approx_b r
		\implies p \approx_b r$.
\end{enumerate}
\begin{enumerate}[leftmargin=*]
\item \textit{Proviamo che $\approx_b$ è riflessiva}. Per il lemma \ref{lemma:11.8.2.1} sappiamo che $Id_Q$ è una bisimulazione di branching; di conseguenza, per definizione di $\approx_b$ risulta che $Id_Q \subseteq\ \approx_b$. Possiamo allora concludere che \[
	\forall\ p \in Q \valeche
	\tupla{p, p} \in Id_Q \subseteq\ \approx_b
	\implies p \approx_b p.
\]
\item \textit{Proviamo che $\approx_b$ è simmetrica}. Per definizione di $\approx_b$, risulta che \[
	\forall\ p,q \in Q \tc p \approx_b q \valeche
		\exists\ \text{bisimulazione di branching}\ S \tc \tupla{p,q} \in S.
\]
Per definizione di inversa, abbiamo che \[
	\tupla{p,q} \in S \implies \tupla{q,p} \in S^{-1}.
\]
Per il lemma \ref{lemma:11.8.2.2} sappiamo che anche $S^{-1}$ è bisimulazione di branching e, dunque, per definizione di $\approx_b$ risulta che $S^{-1} \subseteq\ \approx_b$. Possiamo quindi concludere che \[
	\tupla{q,p} \in S^{-1} \subseteq\ \approx_b \implies q \approx_b p.
\]
\item \textit{Proviamo che $\approx_b$ è transitiva}. Consideriamo $p,q,r \in Q \tc p \approx_b q \myland q \approx_b r$. Per definizione di $\approx_b$, risulta che
\begin{align*}
	&p \approx_b q \implies 
		\exists\ \text{bisimulazione di branching}\ S_1 \tc \tupla{p,q} \in S_1; \\
	&q \approx_b r \implies 
		\exists\ \text{bisimulazione di branching}\ S_2 \tc \tupla{q,r} \in S_2.
\end{align*}
Per definizione di composizione di relazioni, abbiamo che \[
	\tupla{p,q} \in S_1 \myland \tupla{q,r} \in S_2
	\implies \tupla{p,r} \in S_1 \cdot S_2.
\]
Per il lemma \ref{lemma:11.8.2.3} sappiamo che anche $S_1 \cdot S_2$ è bisimulazione di branching e, dunque, per definizione di $\approx_b$ risulta che $S_1 \cdot S_2 \subseteq\ \approx_b$. Possiamo quindi concludere che \[
	\tupla{p,r} \in S_1 \cdot S_2 \subseteq\ \approx_b 
	\implies p \approx_b r.
\]
\end{enumerate}
\end{proof}